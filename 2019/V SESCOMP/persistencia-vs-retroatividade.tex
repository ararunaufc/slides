\begin{frame}
        \frametitle{Terminologia}

        \begin{itemize}
            \item Consultas {\em vs.} Atualizações
            \note[item]{{\em Operações:} Todas as operações que podemos realizar numa estrutura realizam alguma tarefa que pode ser classificada entre \underline{obter uma informação} (obter a resposta a uma pergunta) ou \underline{realizar uma alteração na organização}.}
            \note[item]{{\em Operações:} Existem certas operações que podem se comportar de ambas as formas, mas sempre podemos subdividí-las.}
            \item Versão
            \note[item]{{\em Versão:} Cada configuração (ou estado) de uma estrutura de dados após uma operação de Atualização.}
        \end{itemize}
    \end{frame}

    \begin{frame}
        \frametitle{Estruturas de Dados Temporais}

        \begin{itemize}
            \item Efêmeras
            \begin{itemize}[<2->]
                \item Apenas uma versão (a última) está disponível para manipulação.
                \note[item]{{\em Efêmeras:} As estruturas que chamamos de ``Estruturas de Dados''.}
                \note[item]{{\em Efêmeras:} Efêmero significa passageiro, que dura pouco. Qualquer estado da estrutura dura apenas até a próxima atualização.}
                \note[item]{{\em Efêmeras:} Modelo de memória de um computador. Uma vez feito, não se sabe o que estava lá antes.}
            \end{itemize}
            \item Persistentes
            \begin{itemize}[<3->]
                \item Todas as versões da estrutura estão disponíveis para manipulação.
                \item Atualizações em alguma versão {\em não se propagam} para as demais.
                \note[item]{{\em Persistentes:} Se comportam em analogia à visão que a física tem sobre viagem no tempo.}
            \end{itemize}
            \item Retroativas
            \begin{itemize}[<4->]
                \item Todas as versões da estrutura estão disponíveis para manipulação.
                \item Atualizações em alguma versão {\em sempre se propagam} para as demais.
                \note[item]{{\em Retroativas:} Se comportam em analogia à visão que algumas obras de ficção tem sobre viagem no tempo.}
            \end{itemize}
        \end{itemize}
    \end{frame}

    \begin{frame}
        \frametitle{Persistência {\em vs.} Retroatividade}
        
        \begin{block}{Persistência}
            \centering
            \transparent{0}\includegraphics[width=.5\textwidth]{example-image}
            \note[item]{Várias linhas do tempo. Universos paralelos.}
        \end{block}
        
        \begin{itemize}
            \item Operações ganham um novo parâmetro: {\em a versão que será o seu alvo}.
        \end{itemize}

    \end{frame}

    \begin{frame}
        \frametitle{Persistência {\em vs.} Retroatividade}
        
        \begin{block}{Retroatividade}
            \centering
            \transparent{0}\includegraphics[width=.5\textwidth]{example-image}
            \note[item]{Uma única linha do tempo.}
        \end{block}
        
        \begin{itemize}
            \item Operações ganham um novo parâmetro: {\em o ``momento'' onde serão realizadas}.
        \end{itemize}

    \end{frame}


    \begin{frame}
        \frametitle{Níveis de persistência}

        \begin{block}{Persistência parcial}
            \centering
            \transparent{0}\includegraphics[width=.5\textwidth]{example-image}
        \end{block}

        \begin{itemize}
            \item<1-> Consultas podem ser feitas sobre {\em qualquer versão}.
            \item<2-> Atualizações só podem ser feitas sobre {\em a versão mais recente}.
            \begin{itemize}
                \item<3-> Isso obriga uma distribuição totalmente ordenada das versões.
            \end{itemize}
        \end{itemize}

        \note[item]{Corresponde à forma de percepção de tempo que temos hoje. Lembramos como as coisas eram no passado, mas não podemos alterá-las.}
        \note[item]{De uma certa forma, seria como se só pudéssemos viajar ao passado para sermos ``expectadores''. Ou ainda, corresponderia às versões de ficção com fatos rígidos.}
    \end{frame}

    \begin{frame}
        \frametitle{Níveis de persistência}

        \begin{block}{Persistência total}
            \centering
            \transparent{0}\includegraphics[width=.5\textwidth]{example-image}
        \end{block}

        \begin{itemize}
            \item<1-> Consultas podem ser feitas sobre {\em qualquer versão}.
            \item<2-> Atualizações podem ser feitas sobre {\em qualquer versão}.
            \item<3-> Atualizações não afetam versões já existentes, o que obriga uma ``ramificação'' entre as versões.
            \begin{itemize}
                \item<4-> Isso obriga uma distribuição das versões como uma árvore.
            \end{itemize}
        \end{itemize}

        \note[item]{Corresponde à forma de viagem no tempo de ``universos paralelos''.}
    \end{frame}

    \begin{frame}
        \frametitle{Níveis de persistência}

        \begin{block}{Persistência confluente}
            \centering
            \transparent{0}\includegraphics[width=.5\textwidth]{example-image}
        \end{block}

        \begin{itemize}
            \item Persistência total +
            \item<2-> É possível ``mesclar'' duas ou mais versões em uma nova versão.
        \end{itemize}

        \note[item]{Difícil de imaginar um correspondente quanto a viagem no tempo.}
        \note[item]{Parece haver uma série de televisão antiga que faz uso desse conceito: {\em Sliders}.}
        \note[item]{Corresponde à forma como funcionam os sistemas de controle de versão.}
    \end{frame}

    \begin{frame}
        \frametitle{Níveis de persistência}

        \begin{block}{Persistência funcional}
            \centering
            \transparent{0}\includegraphics[width=.5\textwidth]{example-image}
        \end{block}

        \begin{itemize}
            \item \underline{Nada} pode ser alterado. Toda modificação deve preservar a memória anterior intacta.
        \end{itemize}

        \note[item]{Difícil de imaginar um correspondente quanto a viagem no tempo.}
        \note[item]{Se inspira na forma de comportamento de linguagens funcionais.}
        \note[item]{Se diferencia das demais, pois anteriormente poderíamos alterar parte da memória, desde que o acesso e a interpretação das versões anteriores fosse preservado.}
        \note[item]{Inclui a capacidade de todas as anteriores, já que como nada é alterado, nada é perdido.}
    \end{frame}

    \begin{frame}
        \frametitle{Níveis de retroatividade}

        \begin{block}{Retroatividade parcial}
            \centering
            \transparent{0}\includegraphics[width=.5\textwidth]{example-image}
        \end{block}

        \begin{itemize}
            \item<1-> Atualizações podem ser adicionadas ou removidas em qualquer ``momento'' no passado.
            \item<4-> Consultas só podem ter como alvo o ``momento presente''.
        \end{itemize}

        \note[item]{Seria como mandar alguém para o passado e manter a consciência de como o presente se altera.}
        \note[item]{A física moderna não permite conceber esse tipo de viagem no tempo.}
        \note[item]{Podem acontecer ``paradoxos''.}
        \note[item]{MIB 3}
    \end{frame}

    \begin{frame}
        \frametitle{Níveis de retroatividade}

        \begin{block}{Retroatividade total}
            \centering
            \transparent{0}\includegraphics[width=.5\textwidth]{example-image}
        \end{block}

        \begin{itemize}
            \item<1-> Atualizações podem ser adicionadas ou removidas em qualquer ``momento'' no passado.
            \item<2-> Consultas podem ser adicionadas ou removidas em qualquer ``momento'' no passado.
            \begin{itemize}
                \item<3-> Alterações têm um impacto muito maior (reação em cadeia).
            \end{itemize}
        \end{itemize}

        \note[item]{Seria como você ir para o passado e interferir com o andamento dos acontecimentos.}
        \note[item]{Podem acontecer ``paradoxos''.}
    \end{frame}